\documentclass[11pt]{scrartcl}
\usepackage[T1]{fontenc}
\usepackage[a4paper, left=3cm, right=2cm, top=2cm, bottom=2cm]{geometry}
\usepackage[activate]{pdfcprot}
\usepackage[ngerman]{babel}
\usepackage[parfill]{parskip}
\usepackage[utf8]{inputenc}
\usepackage{kurier}
\usepackage{amsmath}
\usepackage{amssymb}
\usepackage{xcolor}
\usepackage{epstopdf}
\usepackage{txfonts}
\usepackage{fancyhdr}
\usepackage{graphicx}
\usepackage{prettyref}
\usepackage{hyperref}
\usepackage{eurosym}
\usepackage{setspace}
\usepackage{units}
\usepackage{eso-pic,graphicx}
\usepackage{icomma}

\definecolor{darkblue}{rgb}{0,0,.5}
\hypersetup{pdftex=true, colorlinks=true, breaklinks=false, linkcolor=black, menucolor=black, pagecolor=black, urlcolor=darkblue}



\setlength{\columnsep}{2cm}


\newcommand{\arcsinh}{\mathrm{arcsinh}}
\newcommand{\asinh}{\mathrm{arcsinh}}
\newcommand{\ergebnis}{\textcolor{red}{\mathrm{Ergebnis}}}
\newcommand{\fehlt}{\textcolor{red}{Hier fehlen noch Inhalte.}}
\newcommand{\betanotice}{\textcolor{red}{Diese Aufgaben sind noch nicht in der Übung kontrolliert worden. Es sind lediglich meine Überlegungen und Lösungsansätze zu den Aufgaben. Es können Fehler enthalten sein!!! Das Dokument wird fortwährend aktualisiert und erst wenn das \textcolor{black}{beta} aus dem Dateinamen verschwindet ist es endgültig.}}
\newcommand{\half}{\frac{1}{2}}
\renewcommand{\d}{\, \mathrm d}
\newcommand{\punkte}{\textcolor{white}{xxxxx}}
\newcommand{\p}{\, \partial}
\newcommand{\dd}[1]{\item[#1] \hfill \\}

\renewcommand{\familydefault}{\sfdefault}



\newcommand{\themodul}{Probeklausur CAD-CAM}
\newcommand{\thetutor}{Stellberg}

\pagestyle{fancy}
\fancyhead[L]{\footnotesize{C. Hansen}}
\chead{\thepage}
\rhead{}
\lfoot{}
\cfoot{}
\rfoot{}

\title{\themodul{}}
\publishers{\thetutor}


\author{Christoph Hansen \\ {\small \href{mailto:uni@christophhansen.eu}{uni@christophhansen.eu}} }

\date{}

\usepackage{paralist}
\begin{document}

\maketitle

Dieser Text ist unter der
\href{http://creativecommons.org/licenses/by-nc/4.0/}{Creative Commons CC BY-NC 4.0}
Lizenz veröffentlicht.

\textcolor{red}{%
    Ich erhebe keinen Anspruch auf Vollständigkeit oder Richtigkeit. Falls ihr
    Fehler findet oder etwas fehlt, dann meldet euch bitte über den
    Emailkontakt.
}


\hfill\\
\hfill\\
\hfill\\
\hfill\\
\hfill\\
\hfill\\
\hfill\\




Ich halte die Klausur für das SS14 nicht angemessen, da große Teile der Klausur nicht in der Vorlesung behandelt wurden.

\newpage


\section{Frage 1}

Ebene erstellen, Skizze machen und bemaßen


\section{Frage 2}

Ermöglicht paralleles Konstruieren mit gleichzeitiger CAM Umsetzung

\section{Frage 3}

Geometrische Bedingungen und Maße erstellen

\section{Frage 4}

\subsection*{Beispiel 1}

Abstand zweier Linien nach vorgegebenem Maß $\Rightarrow$ Skizze dazu

\section{Frage 5}

Die Mittellinie ist nicht als Kurve geschlossen und man bekäme die Fehlermeldung, dass das Bauteil sich selbst schneidet.

\section{Frage 6}

Die Mittellinie als Rotationsachse definieren und den unteren Teil löschen, da sich das Bauteil sonst wieder selbst scheidet.

\section{Frage 7}

Aus mehreren Körpern mit evtl. nötigen boolschen Operationen

\section{Frage 8}

Instanzen werden beim Zusammenbau verwendet und sind lediglich ein Verweis auf die Originaldatei und sparen bei Mehrfachverbau Datenvolumen.

\section{Frage 9}

Die selben wie für alle anderen Platzierbaren Parts. Also entweder über den Kompass oder über Bedingungen zu anderen Parts.

\section{Frage 10}

Einzelbilder erstellen $\Rightarrow$ kinematische Simulation


\section{Frage 11}

??? nicht in Vorlesung geklärt. Wer Ahnung hat bitte melden!!

\section{Frage 12}

Was soll man da ankreuzen???


\section{Frage 13}

Man nimmt ein 4-seitig berandetes Flächenelement mit isoparametrischen Linien und schneidet daraus ein Dreieck aus.

\section{Frage 14}

Direktschnittstellen und neutrales Datenformat Schnittstellen

\section{Frage 15}

IGES = Initial Graphics Exchange Specification
STEP = Standard for the exchange of product model data

\section{Frage 16}

Beim Rapid Prototyping wird ein Volumenmodell mittel 3D Drucker erstellt. Eine Methode ist z.B das Lasersintern.


ànderung zum Testen von giteye



\end{document}
